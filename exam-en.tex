\def\issolution{1}

\ifdefined\issolution
    \documentclass[10pt,english,solution]{exam-uoc}
\else
    \documentclass[10pt,english]{exam-uoc}
\fi

\begin{document}

\thispagestyle{empty}

\setserie{1}

\lectureheader{G. Arampatzis}{}{Probability Theory}{Spring 2025}

\header{June Exam}{06.06.2025}

\begin{question}{Coins}
%
In a box there are $n+1$ coins.
Coin $i$ shows heads with probability $\frac{i}{n}, \, i=0,\ldots,n$.
You choose a coin at random and flip it. 

\begin{solution}
%
Let $A_i$ be the event of choosing coin $i$ and $K$ be the event of getting heads.
It is given that $\Prob(K \,|\, A_i)=\frac{i}{n}$ and $\Prob(A_i) = \frac{1}{n+1}$.
%
\end{solution}


\begin{subquestion}[10]
%
Show that the probability of getting heads is $\frac{1}{2}$.
%
\end{subquestion}

\begin{solution}
%
Since the events $A_i$ are disjoint and their union gives us the sample space, by the law
of total probability we have,
%
%
\begin{equation}
    \Prob(K) = \sum_{i=0}^n \Prob(K \,|\, A_i) \Prob(A_i) = 
    \sum_{i=0}^n \frac{i}{n} \frac{1}{n+1} =
    \frac{1}{n(n+1)} \frac{n(n+1)}{2} = \frac{1}{2} \,.
\end{equation}
%
%
\end{solution}

\begin{subquestion}[10]
%
Given that heads came up, calculate the probability that you chose coin $i$.
%
\end{subquestion}

\begin{solution}
%
By Bayes' theorem,
%
\begin{equation}
    \Prob( A_i \,|\, K) 
    =
    \frac{\Prob(K \,|\, A_i) \, \Prob( A_i ) }{\Prob( K )}
    =
    \frac{\frac{i}{n} \, \frac{1}{n+1}}{\frac{1}{2}}
    =
    \frac{2i}{n(n+1)} \,.
\end{equation}
%

%
\end{solution}

\end{question}


\begin{question}[20]{Uniform Distribution}
%
Let $X_1, X_2, X_3$ be independent random variables that follow the uniform 
distribution on the interval $[0, 1]$. The probability density function of the uniform 
distribution equals 1 for values of $X$ in $[0, 1]$ and 0 elsewhere.
Calculate the probability that the largest of the three is smaller than the 
sum of the other two.

\begin{solution}
%
The event whose probability we want to calculate can be written as,
%
\begin{equation}
\begin{split}
    A 
    &= \left\{ X_1 < X_2 + X_3\,,\; X_1 > X_2\,,\; X_1>X_3 \right\} \cup 
    \\
    & \quad\, \left\{ X_2 < X_1 + X_3\,,\; X_2 > X_1\,,\; X_2 > X_3 \right\} \cup 
    \\
    & \quad\, \left\{ X_3 < X_1 + X_2\,,\; X_3 > X_1\,,\; X_3 > X_2 \right\} \\
    &=
    A_1 \cup A_2 \cup A_3 \,.
\end{split}
\end{equation}
%
Since the sets are disjoint, we have
%
\begin{equation}
    \Prob(A) = 
    \Prob(A_1) + \Prob(A_2) + \Prob(A_3) =
     3\Prob(A_1)\,,
\end{equation}
%
where the last equality holds due to the symmetry of the problem.
We will calculate the probability of event $A_1$.

For each of the random variables, we have that the largest of all, $X_1$ here, can
belong anywhere in $[0,1]$.  
$X_2$ is certainly smaller than $X_1$ and greater than 0, so it belongs to the 
interval $[0, X_1]$. 
For $X_3$ we have that $X_3 > X_1 - X_2$ and $X_3<X_1$, so it belongs to the interval 
$[X_1 - X_2, X_1]$.
Therefore, 
%
%
\begin{equation}
    \Prob(X_1 < X_2 + X_3)
    = \int_0^1 \int_0^{x_1} \int_{x_1 - x_2}^{x_1} 1 \dif x_3 \dif x_2 \dif x_1
    = \ldots = \frac{1}{6} \,.
\end{equation}
%
Therefore, $\Prob(A) = \frac{1}{2}$.
%
\end{solution}

\end{question}


\newpage\thispagestyle{empty}

\begin{information}
    \begin{itemize}
        \item A total of \gettotalpoints{} points are given. The maximum score is 40.
        \item The duration of the exam is 150 minutes.
        \item Justify your answers clearly. Answers without justification
              are not considered correct.
        \item Solve the problems on draft paper and present the final solutions 
              neatly written. Answers with scribbles will not be graded.
        \item Give consolidated answers for each exercise. 
              If you proceed to the next without having completed the previous one, 
              leave sufficient blank space in case you want to return.
    \end{itemize}
\end{information}


\end{document} 