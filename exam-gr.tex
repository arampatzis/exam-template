\def\issolution{1}

\ifdefined\issolution
    \documentclass[10pt,greek,solution]{exam-uoc}
\else
    \documentclass[10pt,greek]{exam-uoc}
\fi

\begin{document}

\thispagestyle{empty}

\setserie{1}

\lectureheader{Γ. Αραμπατζής}{}{Θεωρία Πιθανοτήτων}{Εαρινό 2025}

\header{Διαγώνισμα Ιουνίου}{06.06.2025}

\begin{question}{Νομίσματα}
%
Σε ένα κουτί υπάρχουν $n+1$ νομίσματα.
Το νόμισμα $i$ φέρνει κορώνα με πιθανότητα $\frac{i}{n}, \, i=0,\ldots,n$.
Διαλέγετε ένα νόμισμα στην τύχη και το στρίβετε. 

\begin{solution}
%
Έστω $A_i$ το ενδεχόμενο να διαλέξω το κέρμα $i$ και $K$ το ενδεχόμενο να έρθει κορώνα.
Είναι δεδομένο ότι $\Prob(K \,|\, A_i)=\frac{i}{n}$ $\Prob(A_i) = \frac{1}{n+1}$.
%
\end{solution}


\begin{subquestion}[10]
%
Να δείξετε ότι η πιθανότητα να έρθει κορώνα είναι $\frac{1}{2}$.
%
\end{subquestion}

\begin{solution}
%
Εφόσον τα $A_i$ είναι ξένα και η ένωση τους μας δίνει το δειγματικό χώρο, από το θεώρημα
της ολικής πιθανότητας ισχύει ότι,
%
%
\begin{equation}
    \Prob(K) = \sum_{i=0}^n \Prob(K \,|\, A_i) \Prob(A_i) = 
    \sum_{i=0}^n \frac{i}{n} \frac{1}{n+1} =
    \frac{1}{n(n+1)} \frac{n(n+1)}{2} = \frac{1}{2} \,.
\end{equation}
%
%
\end{solution}

\begin{subquestion}[10]
%
Με δεδομένο ότι ήρθε κορώνα, να υπολογίσετε την πιθανότητα να διαλέξατε το κέρμα $i$.
%
\end{subquestion}

\begin{solution}
%
Από το θεώρημα του \en Bayes\gr,
%
\begin{equation}
    \Prob( A_i \,|\, K) 
    =
    \frac{\Prob(K \,|\, A_i) \, \Prob( A_i ) }{\Prob( K )}
    =
    \frac{\frac{i}{n} \, \frac{1}{n+1}}{\frac{1}{2}}
    =
    \frac{2i}{n(n+1)} \,.
\end{equation}
%

%
\end{solution}

\end{question}


\begin{question}[20]{Ομοιόμορφη κατανομή}
%
Έστω $X_1, X_2, X_3$ ανεξάρτητες τυχαίες μεταβλητές που ακολουθούν την ομοιόμορφη 
κατανομή στο διάστημα $[0, 1]$. Η συνάρτηση πυκνότητας πιθανότητας της ομοιόμορφης 
κατανομής είναι ίση με 1 για τιμές της $X$ στο $[0, 1]$ και 0 αλλού.
Να υπολογίσετε την πιθανότητα η μεγαλύτερη από τις τρεις να ειναι μικρότερη από το 
άθροισμα των άλλων δύο.


\begin{solution}
%
Το ενδεχόμενο του οποίου την πιθανότητα θέλουμε να υπολογίσουμε γράφεται ως,
%
\begin{equation}
\begin{split}
    A 
    &= \left\{ X_1 < X_2 + X_3\,,\; X_1 > X_2\,,\; X_1>X_3 \right\} \cup 
    \\
    & \quad\, \left\{ X_2 < X_1 + X_3\,,\; X_2 > X_1\,,\; X_2 > X_3 \right\} \cup 
    \\
    & \quad\, \left\{ X_3 < X_1 + X_2\,,\; X_3 > X_1\,,\; X_3 > X_2 \right\} \\
    &=
    A_1 \cup A_2 \cup A_3 \,.
\end{split}
\end{equation}
%
Επειδή τα σύνολα είναι ξένα μεταξύ τους, ισχύει ότι
%
\begin{equation}
    \Prob(A) = 
    \Prob(A_1) + \Prob(A_2) + \Prob(A_3) =
     3\Prob(A_1)\,,
\end{equation}
%
όπου η τελευταία ισότητα ισχύει λόγω συμμετρίας του προβλήματος.
Θα υπολογίσουμε την πιθανότητα του ενδεχομένοNυ $A_1$.

Για κάθε μία από τις τ.μ.\ έχουμε ότι η μεγαλύτερη από όλες, η $X_1$ εδώ, μπορεί να
ανήκει οπουδήποτε στο $[0,1]$.  
Η $X_2$ είναι σίγουρα μικρότερη από την $X_1$ και μεγαλύτερη από το 0, άρα ανήκει στο 
διάστημα $[0, X_1]$. 
Για την $X_3$ έχουμε ότι $X_3 > X_1 - X_2$ και $X_3<X_1$, άρα ανήκει στο διάστημα 
$[X_1 - X_2, X_1]$.
Επομένως, 
%
%
\begin{equation}
    \Prob(X_1 < X_2 + X_3)
    = \int_0^1 \int_0^{x_1} \int_{x_1 - x_2}^{x_1} 1 \dif x_3 \dif x_2 \dif x_1
    = \ldots = \frac{1}{6} \,.
\end{equation}
%
Επομένως, $\Prob(A) = \frac{1}{2}$.
%
\end{solution}

\end{question}


\newpage\thispagestyle{empty}

\begin{information}
    \begin{itemize}
        \item Δίνονται συνολικά \gettotalpoints{} μονάδες. Το μέγιστο σκορ είναι 40.
        \item Η διάρκεια του διαγωνίσματος είναι 150 λεπτά.
        \item Αιτιολογήστε καθαρά τις απαντήσεις σας. Απαντήσεις χωρίς αιτιολόγηση
              δεν θεωρούνται σωστές.
        \item Λύστε τα θέματα στο πρόχειρο και παρουσιάστε τις τελικές λύσεις 
              καθαρογραμμένες. Απαντήσεις με μουντζούρες δεν θα βαθμολογούνται.
        \item Δώστε συγκεντρωμένες απαντήσεις για κάθε άσκηση. 
              Αν προχωρήσετε στην επόμενη χωρίς να έχετε ολοκληρώσει την προηγούμενη, 
              αφήστε επαρκή κενό χώρο σε περίπτωση που θέλετε να επιστρέψετε.
    \end{itemize}
\end{information}


\end{document}